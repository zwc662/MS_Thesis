% This file contains all the necessary setup and commands to create
% the preliminary pages according to the buthesis.sty option.

\title{Safety-Aware Apprenticeship Learning}

\author{Weichao Zhou}

% Type of document prepared for this degree:
%   1 = Master of Science thesis,
%   2 = Doctor of Philisophy dissertation.
%   3 = Master of Science thesis and Doctor of Philisophy dissertation.
\degree=1

\prevdegrees{B.S., Fudan University, 2015}

\department{Department of Electrical and Computer Engineering}

% Degree year is the year the diploma is expected, and defense year is
% the year the dissertation is written up and defended. Often, these
% will be the same, except for January graduation, when your defense
% will be in the fall of year X, and your graduation will be in
% January of year X+1
\defenseyear{2018}
\degreeyear{2018}

% For each reader, specify appropriate label {First, Second, Third},
% then name, and title. IMPORTANT: The title should be:
%   "Professor of Electrical and Computer Engineering",
% or similar, but it MUST NOT be:
%   Professor, Department of Electrical and Computer Engineering"
% or you will be asked to reprint and get new signatures.
% Warning: If you have more than five readers you are out of luck,
% because it will overflow to a new page. You may try to put part of
% the title in with the name.
\reader{First}{Wenchao Li}{Assistant Professor of Electrical and Computer Engineering}
\reader{Second}{Brian Kulis}{Assistant Professor of Electrical and Computer Engineerings}
\reader{Third}{Yannis Paschalidis}{Professor of Electrical and Computer Engineering}

% The Major Professor is the same as the first reader, but must be
% specified again for the abstract page. Up to 4 Major Professors
% (advisors) can be defined. 
\numadvisors=1
\majorprof{Wenchao Li}{{Assistant Professor of Electrical and Computer Engineering}}
%\majorprofb{First M. Last, PhD}{{Professor of Computer Science}}
%\majorprofc{First M. Last, PhD}{{Professor of Astronomy}}
%\majorprofd{First M. Last, PhD}{{Professor of Biomedical Engineering}}

%%%%%%%%%%%%%%%%%%%%%%%%%%%%%%%%%%%%%%%%%%%%%%%%%%%%%%%%%%%%%%%%  

%                       PRELIMINARY PAGES
% According to the BU guide the preliminary pages consist of:
% title, copyright (optional), approval,  acknowledgments (opt.),
% abstract, preface (opt.), Table of contents, List of tables (if
% any), List of illustrations (if any). The \tableofcontents,
% \listoffigures, and \listoftables commands can be used in the
% appropriate places. For other things like preface, do it manually
% with something like \newpage\section*{Preface}.

% This is an additional page to print a boxed-in title, author name and
% degree statement so that they are visible through the opening in BU
% covers used for reports. This makes a nicely bound copy. Uncomment only
% if you are printing a hardcopy for such covers. Leave commented out
% when producing PDF for library submission.
%\buecethesistitleboxpage

% Make the titlepage based on the above information.  If you need
% something special and can't use the standard form, you can specify
% the exact text of the titlepage yourself.  Put it in a titlepage
% environment and leave blank lines where you want vertical space.
% The spaces will be adjusted to fill the entire page.
\maketitle
\cleardoublepage

% The copyright page is blank except for the notice at the bottom. You
% must provide your name in capitals.
\copyrightpage
\cleardoublepage

% Now include the approval page based on the readers information
%\approvalpage
%\cleardoublepage

% Here goes your favorite quote. This page is optional.
%\newpage
%\thispagestyle{empty}
%\phantom{.}
\vspace{4in}

\begin{singlespace}
\begin{quote}
  \textit{Facilis descensus Averni;}\\
  \textit{Noctes atque dies patet atri janua Ditis;}\\*
  \textit{Sed revocare gradum, superasque evadere ad auras,}\\
  \textit{Hoc opus, hic labor est.}\hfill{Virgil (from Don's thesis!)}
\end{quote}
\end{singlespace}

% \vspace{0.7in}
%
% \noindent
% [The descent to Avernus is easy; the gate of Pluto stands open night
% and day; but to retrace one's steps and return to the upper air, that
% is the toil, that the difficulty.]

%\cleardoublepage

% The acknowledgment page should go here. Use something like
% \newpage\section*{Acknowledgments} followed by your text.
\newpage
\section*{\centerline{Acknowledgments}}
I want to express my deepest appreciation to my advisor, Professor Wenchao Li.
During my two-year master's studies, he has always been a great mentor for me. 
He set an example of excellence as a researcher, instructor and role model.
My experiences in his group have always been rewarding. 
Conducting researches under his guidance motivates me to refine my skills and set a high standard for myself. 
I cherish this opportunity for professional and personal development that he has provided me.

I would like to thank my thesis committee members for providing me support through this process. 
Your suggestions and feedback have been invaluable for me.

I also want to thank our department for providing me with such an open and resourceful environment. 
I'm so glad that I will be able to continue enjoying my time here as a PhD student.

Last but not least, I can not tell how grateful I am to my parents.
I would not have been here without your encouragement. You believe in me and always want the best for me.
You are oceans away, yet your love is close.

\vskip 1in

\cleardoublepage

% The abstractpage environment sets up everything on the page except
% the text itself.  The title and other header material are put at the
% top of the page, and the supervisors are listed at the bottom.  A
% new page is begun both before and after.  Of course, an abstract may
% be more than one page itself.  If you need more control over the
% format of the page, you can use the abstract environment, which puts
% the word "Abstract" at the beginning and single spaces its text.

\begin{abstractpage}
% ABSTRACT
It is well acknowledged in the AI community that finding a good reward function for reinforcement learning is extremely challenging. Apprenticeship learning (AL) is a class of ``learning from demonstration'' techniques where the reward function of a Markov Decision Process (MDP) is unknown to the learning agent and the agent uses inverse reinforcement learning (IRL) methods to recover expert policy from a set of expert demonstrations. However, as the agent learns exclusively from observations, given a constraint on the probability of the agent running into unwanted situations, there is no verification, nor guarantee, for the learnt policy on the satisfaction of the restriction. In this dissertation, we study the problem of how to guide AL to learn a policy that is inherently safe while still meeting its learning objective. By combining formal methods with imitation learning, a Counterexample-Guided Apprenticeship Learning algorithm is proposed. We consider a setting where the unknown reward function is assumed to be a linear combination of a set of state features, and the safety property is specified in Probabilistic Computation Tree Logic (PCTL). By embedding probabilistic model checking inside AL, we propose a novel {\it counterexample-guided} approach that can ensure both safety and performance of the learnt policy. This algorithm guarantees that given some formal safety specification defined by probabilistic temporal logic, the learnt policy shall satisfy this specification. We demonstrate the effectiveness of our approach on several challenging AL scenarios where safety is essential.
\end{abstractpage}
\cleardoublepage

% Now you can include a preface. Again, use something like
% \newpage\section*{Preface} followed by your text

% Table of contents comes after preface
\tableofcontents
\cleardoublepage

% If you do not have tables, comment out the following lines
\newpage
\listoftables
\cleardoublepage

% If you have figures, uncomment the following line
\newpage
\listoffigures
\cleardoublepage

% List of Abbrevs is NOT optional (Martha Wellman likes all abbrevs listed)
\chapter*{List of Abbreviations}
\begin{center}
  \begin{tabular}{lll}
    \hspace*{2em} & \hspace*{1in} & \hspace*{4.5in} \\
    AL  & \dotfill & Apprenticeship Learning \\
    IRL   & \dotfill & Inverse Reinforcement Learning \\
    Safe-AL  & \dotfill & Safety-Aware Apprenticeship Learning \\
    CEGIS & \dotfill & Counterexample-Guided Inductive Synthesis \\
    CEX  & \dotfill & Counterexample \\
    PCTL  & \dotfill & Probabilistic Computational Tree Logic \\
    CEGAL  & \dotfill & Counterexample-Guided Apprenticeship Learning \\
  \end{tabular}
\end{center}
\cleardoublepage

% END OF THE PRELIMINARY PAGES

\newpage
\endofprelim
