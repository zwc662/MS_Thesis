
Algorithm 1 is guaranteed to converge. Let $inf_t$ be the $t^{th}$ assigned value of $inf$. After $inf_t$ is given, $k$ is decreased from $k_0=sup$ iteratively by $k_{i}=\alpha\cdot inf_t + (1 - \alpha)k_{i-1}$ until either $|k_i-inf_t|\leq \sigma$ in line 24 or a new safe policy is found in line 18. The update of $k$ satisfies the following equality.
%\vspace{-3mm}
\begin{eqnarray}
\frac{|k_{i+1} - inf_t|}{|k_i - inf_t|}&=& \frac{\alpha\cdot inf_t + (1 - \alpha)k_i - inf_t}{k_i - inf_t} =  1-\alpha
%&=& \frac{k_{i+1} - k^*}{k_i - k^*}\\
\end{eqnarray}
%\vspace{-3mm}

Thus, it takes no more than 1+$\log_{1-\alpha}\frac{\sigma}{sup - inf_t}$ iterations for either the algorithm to converge in line 24 or a new safe policy to be found in line 18. If a new safe policy is found in line 18, $inf$ will be assigned in line 22 by the current value of $k$ as $inf_{t+1}=k$ which obviously satisfies $inf_{t+1} - inf_t \geq(1-\alpha)\sigma$. After the assignment of $inf_{t+1}$, the iterative update of $k$ resumes. Since $sup-inf_t \leq 1$, the following inequality holds.
%\vspace{-3mm}
\begin{eqnarray}
\frac{|inf_{t+1} - sup|}{|inf_{t} - sup|}&\leq&\frac{sup -inf_{t} - (1-\alpha)\sigma}{sup- inf_{t}}\leq 1 - (1-\alpha)\sigma
\end{eqnarray}
%\vspace{-3mm}

Obviously, starting from an initial $inf=inf_0<sup$, with the alternating update of $inf$ and $k$, $inf$ will keep getting close to $sup$ unless the algorithm converges as in line 24 or a safe policy {$\epsilon$-close} to $\pi_E$ is found as in line 19. The extreme case is that finally $inf=sup$ after no more than $\frac{sup-inf_0}{(1-\alpha)\sigma}$ updates on $inf$. Then, the problem becomes AL. Therefore, the worst case of this algorithm can have two phases. In the first phase, $inf$ increases from $inf=0$ to $inf=sup$. Between each two consecutive updates $(t, t+1)$ on $inf$, there are no more than $\log_{1-\alpha}\frac{(1-\alpha)\sigma}{sup - inf_t}$ updates on $k$ before $inf$ is increased from $inf_t$ to $inf_{t+1}$. Overall, this phase takes no more than
%\vspace{-3mm}
\begin{equation}
{\underset{0\leq i< \frac{sup-inf_0}{(1-\alpha)\sigma}}{\sum}\log_{1-\alpha} \frac{(1-\alpha)\sigma}{sup-inf_0-i\cdot (1-\alpha)\sigma}} = {\underset{0\leq i< \frac{1}{(1-\alpha)\sigma}}{\sum}\log_{1-\alpha} \frac{(1-\alpha)\sigma}{1-i\cdot (1-\alpha)\sigma}}
\end{equation}
%\vspace{-3mm}

\noindent iterations to reduce the multi-objective optimization problem to original apprenticeship learning and then the second phase begins. Since $k=sup$, the iteration will stop immediately when an unsafe policy is learnt as in line 24. This phase will not take more iterations than original AL algorithm does to converge.

The convergence analysis of AL can be found in~\cite{Abbeel:2004:ALV:1015330.1015430}. First it's assumed that the ball with radius $\epsilon$ around $\mu_E$, which is a set $\{\mu\mid d(\mu_E, \mu)\leq\epsilon\}$, will have intersection with be the convex hull of the set of feature expectations of candidate policy set $\Pi$ after $n$ iterations. Then it's provable that $n$ satisfies
\begin{equation}
n\leq O\big(\frac{k}{(1-\gamma)^2\epsilon^2}\log{\frac{k}{(1-\gamma)\epsilon}}\big)
\end{equation}
where $k$ is the number of features, $\gamma$ is the discount factor. It's also ensured in~\cite{Abbeel:2004:ALV:1015330.1015430} that the number of sample demonstrations needed to make AL algorithm output with probability at least $1-\delta$ should suffice that
\begin{equation}
m\geq \frac{2k}{(1-\gamma)^2\epsilon^2}\log{\frac{2k}{\delta}}
\end{equation}

In each iteration of Algorithm 1, the algorithm first solves a second-order cone programming (SOCP) problem (\ref{eq:sec5_1}) to learn a policy. SOCP problems can be solved in polynomial time by interior-point (IP) methods~\cite{Kuo:2004aa}.
PCTL model checking for DTMCs can be solved in time linear in the size of the formula and polynomial in the size of the state space~\cite{Hansson1994}. 
Counterexample generation can be done either by enumerating paths using the $k$-shortest path algorithm or determining a critical subsystem using either a $SMT$ formulation or mixed integer linear programming (MILP)~\cite{Wimmer2012}. For the $k$-shortest path-based algorithm, it can be computationally expensive sometimes to enumerate a large amount of paths (i.e. a large $k$) when $p^*$ is large. This can be alleviated by using a smaller $p^*$ during calculation, which is equivalent to considering only paths that have high probabilities.
%\vspace{-1mm}
